% !TEX root = ../section1.tex
\section{Aromatic Heterocycles}

Heterocycles are aromatic compounds with at least one atom that is other than carbon. We will look at two classes:

\begin{enumerate}

  \item 6 Membered Heterocycles

    E.g. Pyridine

    \img{3-1}

    \begin{itemize}
      \item 5 sp\super{2} hybridised carbon atoms with each contributing 1 electon to the $\pi$ system.
      \item 1 sp\super{2} hybridised nitrogen atom which also contributes to the $\pi$ system.
    \end{itemize}

  \item 5 Membered Heterocycles

    E.g. Pyrrole

    \img{3-2}

    \begin{itemize}
      \item 4 sp\super{2} carbon atoms, each contributing 1 electron to the $\pi$ system.
      \item 1 nitrogen atom contributing 2 electons to the $\pi$ system.
      \item Because the nitrogen lone pair is part of the aromatic ring, it is not available for bonding (e.g. to a proton) therefore pyrrole is not a base.
    \end{itemize}

    Other 5 membered rings:

    \img{3-3}

\end{enumerate}

\subsection{Pyrrole}

Pyrrole is aromatic due to the delocalisation of the nitrogen lone pair into the $\pi$system.
Counteracting this resonance effect is an inductive effect.
Nitrogen is less electronegative than carbon. The resonance effect is dominant however the nitrogen atom has a partial positive charge.
Pyrrole is very reactive towards electrophilic substitution.

C-2 is the favourite position for substitution due to the large number of resonance forms.

\img{3-4}

\subsubsection{Electrophilic Substitutions of Pyrrole}

Some typical electrophilic substitutions of pyrrole include:

\img{3-5}

\subsubsection{Nucleophilic Substitutions of Pyrrole}

This process involves taking the Mannich product and treating with methyl iodide to get an ammonium salt. This salt then undergoes the nucleophilic substitution.

\img{3-6}

To react the nitrogen with an electrophile, it is deprotonated with a strong base resulting in the pyrryl anion.

\img{3-7}

E.g. Retrosynthesis of Tolmetin

\img{3-8}

Synthesis:

\img{3-9}

\subsection{Furan}

\img{3-10}

Oxygen is more electronegative than nitrogen therefore furan is less reactive than pyrrole towards electrophiles (but still more than benzene).
Like pyrrole, reaction with E\super{+} is preferentially at the 2- and 5- positions and it is unstable in strongly acidic conditions.
Furan is less aromatic than pyrrole.

\subsubsection{Electrophilic Substitutions of Furan}

\img{3-11}

\subsubsection{Other Reactions of Furan}

\begin{enumerate}[label=\alph*)]
  \item 2,5-addition to Furan

    \img{3-12}

    In some cases a nucleophile adds to the 5 position on the furan instead of re-aromatisation.

    \img{3-13}

    Mechanism

    \img{3-14}

  \item Nitration

    Occurs by a similar process, addition followed by elimination via pyridine.

    \img{3-15}

  \item Diels-Alder reaction of Furans

    Furans behave as dienes in diels-alder reactions with electron deficit alkenes
    and alkynes.

    \img{3-16}

\end{enumerate}

\subsection{Pyridine}

The inductive effect of nitrogen means that the nitrogen has a slight negative charge and the carbons are left slight positive.
Therefore pyridine is referred to as an electron deficient heterocycle and pyridine is unreactive in electrophilic substitution reactions.\\

Electrophilic substitution normally occurs at C-3 as attack at C2/C4 gives rise to an intermediate where the electronegative nitrogen is deficient (high in energy).
Substitution in pyridine is 10\super{7} times slower than in benzene.\\

Conversion of pyridine to the pyridine N-oxide using \ce{H2O2} enables electrophilic substitution.
However substitution commonly takes place in the C-4 position.

\img{3-17}

If relatively activating substituents are present in the pyridine ring then electrophilic substitution can occur more readily, usually ortho/para to the activating group. E.g. Amino pyridines.

\img{3-18}
\img{3-19}

\subsubsection{Nucleophilic Substitution of Pyridine}

The electron deficient nature of pyridines makes them good in nucleophilic substitution reactions, particularly when the negative charge can be stabilised by the nitrogen.\\

Substitution of a leaving group in the 2 position:

\img{3-20}

In the 3 position:

\img{3-21}

In the 4 position:

\img{3-22}

With attack at C2/C4, the anion is stabilised by the nitrogen:

\img{3-23}

2-alkyl and 4-alkyl pyridines are acidic:

\img{3-24}