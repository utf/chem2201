% !TEX root = ../section2.tex
\section{Polycyclic Aromatic Hydrocarbons}

\subsection{Napthalene}

\img[H\"uckels rule applies here with n=2]{2-1}

The aromatic stability of napthalene is higher than benzene but less than twice as large.
For this reason it is easier to disrupt aromatically and so napthalene is more reactive than benzene in electrophilic substitutions.
E.g. It will react with \ce{Cl2} without a catalyst.

There are 2 points of electrophilic attack, C-1 or C-2:

\img{2-2}

The lowest resonance forms retain aromaticity in the left hand ring. C-1 attack has two such forms, C-2 has only one.
Therefore, the energy of the intermediate from attack at C-1 is lower than C2 so substitution is more rapid at C-1.
In general, in polycyclic aromatic hydrocarbons, electrophilic substitution occurs most readily to a ring junction.
However, the major product in the sulfonation of napthalene depends on the conditions under which the reaction is carried out.

\img{2-3}

This is because at 80 \dec\ the reaction is irreversible and therefore the product obtained is fasted formed (kinetic product).
At 160 \dec\ the reaction is reversible therefore the product formed is the most stable (thermodynamic production).

If we carry out electrophilic substitutions on substituted napthalene:
\begin{itemize}
  \item Substitution will occur in the more electron rich ring
  \item Substitution will occur adjacent to the ring junction
  \item Normal directing effects apply
\end{itemize}

E.g.

\img{2-4}

\subsection{Nucleophilic Substitution in Aromatic Compounds}

Nucleophilic substitution in aromatic compounds is rare as the S\sub{N}2 mechanism is impossible at the sp\super{2} hybridised carbons of benzene.

\img{2-5}

It only occurs under 3 sets of circumstances:
\begin{enumerate}
  \item There is a very good leaving group e.g. \ce{N2} (g)
  \item The nucleophile is also a strong base e.g. \ce{^{--}NH2}
  \item There are strongly electron withdrawing substituents ortho and/or para to the leaving group.
\end{enumerate}

Each of these has a different mechanism.

\begin{enumerate}[label=\alph*)]
  \item S\sub{N}1 Mechanism

    \img{2-6}

    Diazonium formation

    \img{2-7}

    Mechanism

    \img{2-8}

  \item Benzyne Mechanism

    \img{2-9}

    The nucleophile must be a strong base, \ce{HO-} or \ce{^{--}NH2} are most commonly used.

  \item S\sub{N}Ar Mechanism

    \img[Addition of Nu is the rate determining step therefore leaving group ability of X is not important but electronegativity of X is important (F > Cl > Br)]{2-10}

    This mechanism requires strongly electron withdrawing groups ortho and/or para to the leaving group.

    \ce{NO2} is best for this purpose followed by carbonyl groups.

    \img{2-11}

\end{enumerate}

\subsection{Birch Reduction}

This is the partial reduction of aromatic rings.

\img{2-12}

Mechanism:

\img[The product formed depends on which resonance ion is protonated first]{2-13}

If the Birch reduction is carried out with monosubstituted benzene then there are 2 possible products.

\img{2-14}

The isomer obtained depends on the nature of the substituent:
\begin{itemize}
  \item If R is e\super{--} withdrawing (e.g. carbonyl group) isomer A is obtained.
  \item If R is e\super{--} donating (e.g. alkyl or OR), isomer B is obtained.
  \item If there is more than one substituent, it occurs as much as possible to put e\super{--} donating substituents on the double bond and e\super{--} withdrawing ones off them.
\end{itemize}

\img{2-15}

Birch reduction followed by ozonolysis is useful for generating 1,6 dicarbonyl compounds.

\img{2-16}