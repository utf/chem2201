% !TEX root = ../section1.tex
\section{Heterocycle Synthesis}

\subsection{Paal-Knorr Pyrrole Synthesis}

\img{4-1}

So if we react a 1,4 diketone and an amine (with acid) we get a pyrrole.

\img{4-2}

Mechanism

\img{4-3}

If we don't want a substituent on the nitrogen we can use ammonium acetate as a source of \ce{NH3}.

\img{4-4}

\subsection{Paal-Knoor Furan Synthesis}

Via a slight modification to the retrosynthesis we can apply the same method to furans:

\img{4-5}

This is dehyration of a 1,4 dikenone (anhydrous conditions).

\img{4-6}

Mechanism

\img{4-7}

\pagebreak

\subsection{Synthesis of Thiophenes}

Prepared from 1,4 diketone and a source of sulfur such as \ce{H2S}, \ce{P4S10} or Lawessons reagent.

\img{4-8}

Using:

\img{4-9}

With the mechanism:

\img{4-10}

\pagebreak

\subsection{Knorr Pyrrole Synthesis}

If there is an electron withdrawing group in the 3 position of a pyrrole, another disconnection is possible.

\img{4-11}

Therefore $\beta$ keto ester + $\alpha$ amino ketone + base results in a ketone.

\img{4-12}

Mechanism - slightly different to disconnection:

\img{4-13}

\pagebreak

\subsection{Synthesis of Pyridines}

The disconnections similar to those above work but the starting materials are not easy to make and therefore the last step is changed to an oxidation.

\img{4-14}

Oxidation to pyridines:

\img{4-15}

Therefore 1,5 dicarbonyl + \ce{NH3} + oxidation results in a pyridine.

e.g.

\img{4-16}

Alternatively we can make the last step a dehydration rather than oxidation by using hydroxylamine  (\ce{H2N-OH}) rather than ammonia.

\img{4-17}

\pagebreak

\subsection{Hantzsch Pyridine Synthesis}

\img{4-18}

The C3/C5 substituents must be electron withdrawing groups (usually esters).
If the target is unsymmetrical pyridine or dihydropyridine we can seperate the 2 halfs (i.e. keep the enamine seperate from the enone formation).

\img{4-19}

Forward synthesis:

\img{4-20}

Formation of enone and enamine:

\img{4-21}

Combination mechanism:

\img{4-22}
