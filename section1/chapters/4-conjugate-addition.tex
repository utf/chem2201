% !TEX root = ../section1.tex
\section{Conjugate Addition Reactions}

Occurs due to the resonance effects of a $\alpha$,$\beta$-unsaturated carbonyl
compounds.

\img{4-1}

There are two types of conjugate addition:

\img{4-2}

\textbf{(1)} 1,4 Conjugate addition, produces the thermodynamic product which is
more stable but $\Delta$Ea is greater.

\textbf{(2)} 1,2 Conjugate addition, produces the kinetic product, which is
reversible at higher temperatures.\\

Conjugate addition only occurs with alkenes conjugated to a $\pi$ electron
withdrawing group. For example:

\img[This does \textbf{not} occur as there is no way to stabilise the anion]{4-3}

1,4 vs 1,2 addition
\begin{itemize}
  \item 1,4 addition is favoured by less reactive \ce{C=O} groups. 1,2 is favoured
    by more reactive \ce{C=O} groups.
  \item Hard nucleophiles (RMgBr, RLi) react at the \ce{C=O} group. Soft nucleophiles
    (\ce{RS-}, \ce{RCu}) perform conjugated addition.
  \item Steric hindrance at the $\beta$ carbon favours 1,2-addition, an unhindered
    $\beta$ position favours 1,3 addition.
\end{itemize}

\subsection{Nucleophiles for Conjugate Addition}

\begin{enumerate}[label=\alph*)]

  \item Thiols
    \img{4-4}

  \item Amines

    \img{4-5}

    Amines such as aniline can perform 2 conjugate addition reactions by displacing
    the H's. E.g.

    \img{4-6}

  \item Nitrile

    At high temp (80 \dec) 1,4 addition will occur:

    \img{4-7}

    At low temp (5 -- 10 \dec) 1,2 addition will occur:

    \img{4-8}
  \item Alcohols

    Acid catalysed:

    \img{4-9}

    Base catalysed:

    \img{4-10}

    With acid catalysed there is competition with the conjugate addition and forming
    an acetal.

  \item Organometallic Compounds

    Grignard and organolithium reagents can perform 1,2 conjugate addition:

    \img{4-11}

    To make them perform 1,4 conjugate addition, react them with cuprates, e.g.
    CuCl.

    \img{4-12}

    Hard reagents:
    \begin{itemize}
      \item Perform 1,2 conjugate addition

      \item Nucleophiles are small electronegative atoms (O, Cl) or small counter
        ions (\ce{R-}, \ce{Li+})

      \item React with hard electrophile (more explicit charge) by electrostatic
        interaction.
    \end{itemize}

    Soft reagents:
    \begin{itemize}
      \item Perform 1,4 conjugate addition

      \item Nucleophiles are larger atoms (S, I) or less polarised C metal bonds
        \ce{R-Cu}

      \item React with soft electrophiles under orbital control.
    \end{itemize}

  \item Enolates, Enols and Equivalents

  Hard enolates such as \imginline{4-13} react via 1,2 conjugate addition.

  Soft enolates such as \imginline{4-14} which has a more delocalised charge,
  or \imginline{4-15} react via 1,4 conjugate addition.

  E.g.

  \img{4-16}

\end{enumerate}

\subsection{The Robinson Annelation}

The Robinson annelation is the result of a conjugate addition followed by aldol
cyclisation. The requirement for a Robinson annelation is a Michael addition of
an enolate to an enone that has a second enolisable group on the other side of
the ketone.

The first step is the formation of the stable enolate:

\img{4-17}

The second stage is the formation of a new enolate on the other side of the
ketone from the first:

\img{4-18}

The final stage is the dehydration of the aldol and an E1cB reaction that involves
the carbonyl group in a standard aldol reaction. Another enolate must form in the
same position as the last.

\img{4-19}