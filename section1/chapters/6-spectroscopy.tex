% !TEX root = ../section1.tex
\section{Spectroscopy}

\subsection{Summary of IR Frequencies}

\begin{tabbing}

\textbf{Single Bonds to Hydrogen} \\

  ~~~~~ \= - sp\super{3} ~~~~~~~~~~~~~~~~~~~~~~~~~~~~~~~~ \=
    \ce{C-H} ~~~~~~~~~~~~~~~~~~~ \= 2850 -- 2960s \\
  \> - sp\super{2}       \> \ce{C-H}            \> 3010 -- 3095 \\
  \> - Aldehyde          \> \ce{O=C-H}          \> 2700 -- 2900 \\
  \> - sp                \> \ce{C-H}            \> 3000 sharp \\
  \> - Nitrile           \> \ce{N-H}            \> 3300 -- 3500m \\
  \> - Free              \> \ce{O-H}            \> 3590 -- 3600s sharp \\
  \> - Normally H-bonded \> \ce{O-H}            \> 3200 -- 3600s broad \\
  \> - Strongly H-bonded \> \ce{O-H}            \> 2500 -- 3200s broad \\\\\\


\textbf{Triple Bonds} \\

  \> - Nitriles          \> \ce{RCN}            \> 2200 -- 2260v \\
  \> - Alkynes           \> \ce{RC#CR^{1}}      \> 2150 -- 3095w \\
  \>                     \> \ce{RC#CH}          \> 2100 -- 2140w \\\\\\


\textbf{Double Bonds} \\

  \> - Alkenes           \> \ce{C=C}            \> 1620 -- 1680v \\
  \> - Enones            \> \ce{C=C-C=O}        \> 1590 -- 1640s \\
  \> - Aromatics, up to 3 of \>                 \> 1600, 1580, 1500v \\
  \> - Nitro             \> \ce{NO2}            \> 1560, 1350s \\\\\\


\textbf{Carbonyl Group, C=O} \> \> \> 1715 $\pm$ 10 \\

  \> - Dialkyl ketone    \> alkyl               \> 0 (also carboxylic acid is $\pm$ 0)\\
  \> - Anhydride         \> OCOR                \> $+35$, $+110$ \\
  \> - Acid chloride     \> COOCl               \> $+85$ \\
  \> - Ester             \> OCOR                \> $+25$ \\
  \> - Aldehyde          \> H                   \> $+15$ \\
  \> - Aryl ketone       \> Ar                  \> $-25$ \\
  \> - Enone             \> \ce{C=C}            \> $-35$ \\
  \> - Amide             \> \ce{NH2}            \> $-65$ \\
\end{tabbing}

\pagebreak

\subsection{\texorpdfstring{\super{1}H}~ NMR}

Most signals are from 0 -- 12 ppm

\begin{tabbing}
  ~~~~~ \= - Carboxylic acids ~~~~~~~~~~~~~~  \= 10 -- 12 \\
  \> - Aldehydes                          \> 9 -- 10 \\
  \> - Aromatics                          \> 7 -- 9 \\
  \> - Alkenes                            \> 5 -- 7 \\
  \> - \imginline{6-1}                    \> 3 -- 5 \\
  \> - \imginline{6-2}                    \> 2 -- 3.5 \\
  \> - \imginline{6-3}                    \> 2 -- 3 \\
  \> - \imginline{6-4}                    \> 2 -- 3 \\
  \> - Alkynes                            \> 2 -- 3 \\
  \> - Alkanes                            \> 0.5 -- 1.5 \\
\end{tabbing}


\ce{R-OH}, \ce{R-SH}, \ce{R-NH2} are hard to predict. Often 0 -- 5 ppm with little
H bonding. Higher if more H bonding. The more shielded the hydrogen the further
upfield it appears (i.e. lower frequency).

\subsubsection{Summary of Chemical Shifts}

\begin{tabbing}
\textbf{Methyl Groups}\\

  ~~~~~ \= - \ce{CH3-C} ~~~~~~~~~~~~~~~~~~~~~~~~~~  \= 10 -- 12 \\
  \> - \ce{CH3-C=C}                            \> 9 -- 10 \\
  \> - \ce{CH3-Ar}                             \> 7 -- 9 \\
  \> - \ce{CH3-CO-R}                           \> 5 -- 7 \\
  \> - \ce{CH3-O-R}                            \> 3 -- 5 \\
  \> - \ce{CH3-N}                              \> 2 -- 3.5 \\\\\\

\textbf{Protons Attached to Unsaturated Linkages}\\
  \> - \ce{Ar-CHO}                              \> 9.7 -- 10.5 \\
  \> - \ce{RCHO}                                \> 9.4 -- 10.0 \\
  \> - \ce{H-CO-O}                              \> 8.0 -- 8.2 \\
  \> - Aromatic                                 \> 6.0 -- 9.0 (usually $\approx$ 7)\\
  \> - \ce{C=CH-CO}                             \> 5.8 -- 6.7 \\
  \> - \ce{C=CH}                                \> 4.5 -- 6.0 \\
  \> - \ce{C#C-H}                               \> 1.8 -- 3.1 \\
\end{tabbing}


\pagebreak

\subsubsection{J Couplings}

\begin{tabbing}
\textbf{Methyl Groups}\\

  ~~~~~ \= - Open chain single bond ~~~~~ \= 7 Hz \\
  \> - Trans alkene             \> 12 -- 18 Hz (typically 16 Hz) \\
  \> - Cis alkene               \> 7 -- 11 Hz (typically 10 HZ) \\
  \> - Ortho                    \> 6 -- 9 Hz \\
  \> - Meta                     \> 1 -- 3 Hz \\
  \> - Para                     \> 0 -- 1 Hz \\
  \> - Cyclohexane              \> 10 -- 12 Hz \\
  \> - H\sub{ax} H\sub{eq}      \> 3 -- 5 Hz \\
  \> - H\sub{eq} H\sub{eq}      \> 3 -- 4 Hz \\\\\\
\end{tabbing}

If \ce{D2O} is added and a signal disappears, it means they acidic protons e.g.
OH, \ce{NH2}, etc.

If the coupling is to inequivalent protons, coupling constants may not be the
same and double doublets are observed. The major coupling always comes first.

\img{6-5}

Long range coupling occurs. H\sub{A} is split into a doublet with H\sub{B} and
H\sub{B} is further split into a triplet by H\sub{C} forming a double triplet
(or dt for short).

\subsection{\texorpdfstring{\super{13}C, \super{19}F, \super{31}P}~ NMR
Spectroscopy}

If \super{19}F or \super{31}P are present in a sample, coupling can be seen in
\super{1}H NMR spectra.\\

The number of resonances in \super{13}C spectra indicates the number of distinct
\super{13} environments in the molecule. The usual solvent is CDCl\sub{3} and a
peak can be seen sometimes at 77 ppm.\\

Because the \super{13}C nucleus is is isotopically rare, it is unlikely that two
adjacent carbon atoms will be \super{13}C therefore \ce{^{13}C\sbond^{13}C} are
not observed. However \super{13}C does strongly couple to any protons attached.
These couplings are normally removed by irradiating the \super{1}H nuclei during
\super{13} acquisition, resulting in a \super{1}H Decoupled \super{13} spectrum.

\pagebreak

\subsubsection{Summary of positions in \texorpdfstring{\super{13}C}~ NMR}

\begin{tabbing}
\textbf{Methyl Groups: \ce{CH3-X} where the table below shows X}\\

  ~~~~~ \= - \ce{CH3} ~~~~~~~~~~~~~~~~~~~~~~~~~~~~~~ \= 7.3 Hz \\
  \> - \ce{CH2CH3}              \> 15.4 Hz (typically 16 Hz) \\
  \> - Phenyl                   \> 21.4 Hz (typically 10 HZ) \\
  \> - Cl                       \> 25.6 Hz \\
  \> - \ce{NH2}                 \> 28.3 Hz \\
  \> - \ce{COCH3}               \> 30.7 Hz \\
  \> - \ce{OH}                  \> 50.2 Hz \\\\\\

\textbf{Monosubstituted Alkanes: \ce{CH3-CH2-X}}\\

  \> - Phenyl                   \> 34.3 Hz \\
  \> - Cl                       \> 53.7 Hz (typically 16 Hz) \\
  \> - OH                       \> 64.0 Hz (typically 16 Hz) \\\\\\

\textbf{Alkenes \ce{R-CH=CH2}}\\
  \> - base value               \> 123.3 \\
  \> - \ce{OH3}                 \> $+ 294$ \\
  \> - \ce{COCH3}               \> $+ 13.8$ \\\\\\

\textbf{Aromatics}~~~ \imginline{6-6} \\
  \> - base value               \> 128.5 \\
  \> - C1                       \> $+31.4$ \\
  \> - C2                       \> $- 14.4$ \\
  \> - C3                       \> $+ 1.0$ \\
  \> - C4                       \> $- 7.7$ \\\\\\
\end{tabbing}

\subsection{Double Bond Equivalents}

Double bond equivalents (DBE) is the number of double bonds and rings.

\begin{center}
  \ce{C_aH_bO_cN_d} ~~~
$DBE=\dfrac{(2a + 2)-(b-d))}{2}$

\end{center}