% !TEX root = ../section1.tex
\section{Retrosynthetic Analysis}

\subsection{Terminology}

Target molecule (TM) is the final product.

A synthon is an idealised reagent that shows the desired reactivity for the
particular disconnection.

E.g. \imginline{5-1} is a synthon. Both \imginline{5-2} react like it.

A functional group interconversion is transforming one functional group into
another without disconnecting anything. The aim is to make the next disconnection
easier.

\img{5-3}

\subsection{Useful Reactions}

Adding SMe ~~~~~~~~~~~~~

\img{5-4}

Reducing ketones + esters ~~~~~~~~~~~~~

\img{5-5}

Joining a ring with two carboxylic acids ~~~~~~~~~~~~~

\img{5-6}

Hydrogenating \ce{C#N} ~~~~~~~~~~~~~

\img{5-7}

Creating ring with amide and ketone ~~~~~~~~~~~~~

\img{5-8}

Mannich Reaction ~~~~~~~~~~~~~

\img{5-9}

Another example of a functional group interconversion is ~~~~~~~~~~~~~

\img{5-10}