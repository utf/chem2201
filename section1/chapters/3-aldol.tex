% !TEX root = ../section1.tex
\section{Aldol Reaciton}

In mild or dilute base, results in the self condensation of an aldehyde or ketone.

\img{3-1}

In presence of conc base a further reaction can take place

\img[Forms a $\alpha$,$\beta$-unsaturated aldehyde]{3-2}

If an unsymmetrical ketone with more than one $\alpha$-H is used then 2 different
products will be formed. However if the material only has 1 way to enolise, then
the aldol reaction will only form one product.\\

Adol reactions can also occur between two different \ce{C=O} compounds.

\img{3-3}

However other reactions can happen, e.g. it will react with itself and the
other starting product may also enolise. NaOH is therefore not an efficient way
to form a single product. However it is possible to form a single product by
taking into account that:
\begin{itemize}
  \item A compound with no $\alpha$-H cannot enolise
  \item Aldehydes are more reactive than ketones
\end{itemize}
Beware however that if you have a molecule with no $\alpha$-H and it is a ketone
reacting with an aldehyde the ketone will be ignored and a single product will
be formed by the aldehyde reacting with itself.\\

Using acetoacetate as a reagent for \imginline{3-4}

\img{3-5}

Crossed aldol product reaction with only one product using LDA to from lithium
enolates.

\img[Addition of LDA results in complete conversion to lithium enolate]{3-6}

\subsection{Unsymmetrical Ketones}

To remove the less sterically hindered protons, a bulky base is used. E.g. LDA
\@ -78 \dec . If we need the more sterically hindered enolate, we can do this
by acknowledging that this is the more stable enolate and use \ce{Me3SiCl} and
\ce{Et3N} then \ce{MeLi}. E.g.

\img[\ce{Et3N} is used as a mild unhindered base that an remove either protons]
  {3-7}

Aldehydes and LDA will not react to form enolates as \imginline{3-8}
Instead to make an enolate from an aldehyde use cyclohexanamine, \ce{H+} and then
LDA.

\img{3-9}

Formaldehyde is even more reactive than other aldehydes, this makes it impossible
to control as an enol. Formaldehyde is therefore not useful for adding a
\ce{CH2-OH} group to molecules. To convert

\img{3-10}

we can't therefore go through:

\img{3-11}

Instead the Mannich reaction is therefore used:

\img{3-12}

\subsection{Electrophiles}

The choice of electrophile for for enolate alkylation is important:
\begin{itemize}
  \item Enolate alkylation are S\sub{N}2 reactions
  \item \ce{R-X}: X must be a good leaving group
  \item Mesylate > Tosylate > I > Br > Cl
\end{itemize}

\img{3-13}

Example:

\img{3-14}

\subsection{Other Reactions}

Diekman Condensation - Both esters in the same molecule:

\img{3-15}

Crossed Claisen Reactions - Between 2 esters or one ester and a ketone. The
conditions are:
\begin{itemize}
  \item Need one ester that can't enolise
  \item This is ester must be a better electrophile
\end{itemize}

There are only 3 reagents. All are more reactive than simple esters.

\img{3-16}

Example:

\img{3-17}

With Claisen you normally end up with a diketone, e.g.

\img{3-18}

